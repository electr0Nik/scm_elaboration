\chapter{Überblick}
Das erste Kapitel soll einen generellen Überblick über den Aufbau und Ziele der Dokumentation formuliert.
 
\section{Ziel der Dokumentation}
Das Ziel dieser Ausarbeitung ist die Schaffung eines Überblicks über die Module und Funktionalität des SAP APO Systems. Dabei soll das Dokument nicht nur eine Evaluation einer SCM Software darstellen, sondern auch ein Nachschlagewerk für spätere private und/oder geschäftliche Zwecke darstellen.

\section{Aufbau der Dokumentation}
Die Dokumentation ist in fünf große Gebiete unterteilt. In Kapitel zwei steht der Begriff Supply Chain Management in Vordergrund. Dabei wird nicht nur der Bezeichnung erklärt, sondern auch die Voraussetzungen, Ziele und der Nutzen erarbeitet. Kapitel drei stellt die einzelnen Komponenten des SAP APO Systems dar. Hier wird keine detailgenau Übersicht erstellt, sondern nur deren generelle Aufgaben und Funktionen dargestellt. Kapitel vier geht genauer auf die im vorherigen Absatz beschriebenen Elemente ein. Dazu wurde fiktiv ein Unternehmen aufgesetzt und das nötige Customizing betrieben um dieses im SAP System abzubilden. Das letzte Kapitel umfasst eine generelle Reflexion der erstellten Hausarbeit. Hier ist vor allem wichtig das nicht nur das vorher skizzierte Fallbeispiel betrachtet, sondern auch ein Blick über den Tellerrand hinaus gewagt wird.