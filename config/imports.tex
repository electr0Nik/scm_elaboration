% Umlaut unter UTF-8 nutzen
\usepackage[utf8]{inputenc}

% Grafiken aus PNG Dateien
\usepackage{graphicx}

% Hilfspaket für Kopf- und Fusszeile
\usepackage{fancyhdr}

% Deutsche Rechtschreibung, Sonderzeichen und Silbentrennung
\usepackage[ngerman]{babel}

% Sonderzeichen Euro
\usepackage[right]{eurosym}

% Zeichenkodierung
\usepackage[T1]{fontenc}

% flüssigere Darstellung der Schriftart im PDF
\usepackage{lmodern}

% Textfarben
\usepackage{color}

% erstelle Inhaltsverzeichnis mit Querverweisen zu den Abschnitten
\usepackage[bookmarksnumbered, pdftitle={Dokumentation}, hyperfootnotes]{hyperref}
%\hypersetup{colorlinks, citecolor=red, linkcolor=blue, urlcolor=black}
%\hypersetup{colorlinks, citecolor=black, linkcolor= black, urlcolor=black}

% Zeilenabstand 
\usepackage{setspace}

% Zeilenabstand für Bildbezeichner
\usepackage{capt-of}

% Stichwortverzeichnis
\usepackage{makeidx}

% Stil der Zitierng
\usepackage[numbers,round]{natbib} % Runde Klammern

% Abkürzungsverzeichnis
% \usepackage[german]{nomencl}
\usepackage[printonlyused, withpage]{acronym} % Nur benuzte Akronyme ausgeben, mit Seitenzahl des ersten Auftrettens

% Paket für Tabellen
\usepackage{array}

% mehrseitige Tabellen
\usepackage{longtable}

% beliebig rotierbare Bilder ermöglichen
%\usepackage{floatflt}

% Packet für Seitenrandabstände
\usepackage{geometry}

% Paket für Boxen im Text
\usepackage{fancybox}

% fußzeile nummerierung nicht reseten
\usepackage{chngcntr}

% override numbering of sections
\usepackage{titlesec}
% set depth to 4; 1. chapter -> 1.1 section -> 1.1.1 subsection -> 1.1.1.1 subsubsection -> 1.1.1.1.1 paragraph
\setcounter{secnumdepth}{4}
